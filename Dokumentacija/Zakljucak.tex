\chapter{Zaključak i budući rad}
				
		 Zadatak naše grupe bio je razvoj web aplikacije za organiziranje znanstvenih konferencija uz mogućnost prijavljivanja sudionika, učitavanja znanstvenoga rada od strane prijavljenog sudionika, recenziranja učitanih radova od strane recenzenata, pregled radova lokalno i online od strane organizatora konferencije i recenzenata te uz mogućnosti pridijeljenih administratoru.\\
		 Nakon 15 tjedana rada u timu i razvoja, ostvarili smo zadani cilj. Sama provedba projekta bila je gotova kroz dvije faze.\par
		 		 
		 Prva faza projekta uključivala je okupljanje tima za razvoj aplikacije, dodjelu projektnog zadatka i intenzivan rad na dokumentiranju zahtjeva. Kvalitetna provedba prve faze olakšala je daljnji rad na realizaciji aplikacije.
		 Izrađeni su obrasci i dijagrami (obrasci uporabe, sekvencijski dijagrami, model baze podataka i dijagram razreda). Oni su uvelike pomogli podtimovima zaduženim za razvoj \textit{frontenda} i \textit{backenda}. Vizualizacija samih idejnih rješenja pomoću obrazaca i dijagrama je uštedila mnogo vremena u drugom ciklusu u trenutcima kada su članovi tima nailazili na nedoumice oko implementacije rješenja.\par
		 
		 Druga faza projekta je bila nešto kraća, ali i intezivnija po pitanju samostalnog rada svakog člana tima. Na početku druge faze je urađena podjela tima na one koji će raditi na \textit{frontendu}, one koji će raditi na \textit{backendu} i na one koji će raditi na bazi podataka sukladno njihovim vlastitim željama i iskustvu u tehnologijama u kojima se projekt radio. Osim realizacije rješenja, u drugoj fazi je bilo potrebno dokumentirati ostale UML dijagrame i izraditi popratnu dokumentaciju kako bi budući korisnici mogli lakše koristiti ili vršiti preinake na sustavu.\par
			
		 Komunikacija među članovima je bila putem aplikacija Whatsapp i Discord čime je olakšano informiranje članova o napretku projekta. Upravljanje izvornim kodom se vršilo sustavom Git, a repozitorij projekta nalazi se na se na web platformi GitLab.\par
		
		 Sudjelovanje na ovakvom projektu bilo je vrijedno iskustvo svim članovima tima jer smo kroz 15 tjedana iskusili zajednički rad na projektima. Spoznali smo važnost dobre vremenske organiziranosti i koordiniranosti među članovima tima. U slučaju da bi semestar tj. vrijeme za izradu bilo produženo za barem još jedan ciklus tj. 5-6 tjedana, aplikacija bi bila dotjerana do savršenstva.\\
		 Zadovoljni smo postignutim bez obzira na neke teškoće i u komunikaciji i u samom razvoju aplikacije te smatramo da smo ovim projektom dobili iskustvo i barem donekle dočarali što nas čeka nakon završetka školovanja.
		\eject 